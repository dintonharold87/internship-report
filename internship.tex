\documentclass{article}
\renewcommand{\baselinestretch}{1.5}
\setcounter{tocdepth}{3}
\usepackage{graphicx}
\usepackage{apalike}

\begin{document}
\begin{titlepage}
\begin{center}
\textbf{MAKERERE UNIVERSITY}
\end{center}
COLLEGE OF COMPUTING AND INFORMATION SCIENCES\\
SCHOOL OF COMPUTING AND INFORMATICS TECHNOLOGY\\
\begin{center}
A REPORT ON\\FIELD ATTACHMENT/INTERNSHIP AT\\NATIONAL WATER \& SEWERAGE CORPORATION\\(June-August,2018)
\end{center}
\begin{center}
BY\\AINEMUKAMA DINTON HAROLD\\16/U/3020/PS
\end{center}
\hangindent 2cm Field attachment Report submitted to the School of Computing and Informatics Technology or College of Computing and Information Sciences.\\
\par In Partial fulfilment of the requirements for the degree of Computer Science of Makerere University Kampala.
\begin{center}
Ainemukama Dinton Harold:\dotfill
\end{center}
\makebox[2.5in]{\hrulefill} \hspace{0.3in}\makebox[2.5in]{\dotfill}\\
Field Supervisor \hspace{1.5in} \makebox[3.0in][r]{Stamp \& Date:\hrulefill}\\
\vspace{.1in}
\makebox[2.5in]{\hrulefill} \hspace{0.3in}\makebox[2.5in]{\dotfill}\\
Academic Supervisor  \hspace{1.5in} \makebox[2.0in][r]{Date:\hrulefill}\\
\end{titlepage}
\newpage
\textbf{Approval of the Internship Report}
\par I Ainemukama Dinton Harold of Registration Number 16/U/3020/PS, sincerely declare that the Internship Report is submitted to the partial fulfilment of the internship program during the last two months. Any part of this report has not been reported or copied from any report of the University and others.\\
Approved by:\\
\makebox[2.5in]{\hrulefill} \hspace{0.3in}\makebox[2.5in]{\dotfill}\\
Field Supervisor \makebox[3.0in][r]{Signature}\\
\vspace{.1in}
\makebox[2.5in]{\hrulefill} \hspace{0.3in}\makebox[2.5in]{\dotfill}\\
Academic Supervisor \makebox[2.5in][r]{Signature}\\
\newpage
\textbf{Acknowledgment}\\
The special thanks goes to my helpful supervisor \textbf{Mr.Willy Nuwamanya},(Senior Manager Billing \& IT). The supervision and support that he gave truly helped in the progression and smoothness of the internship program.
\par I would also like to thank \textbf{Mr.Ivan Atwine}(Senior Systems Administrator Kampala Water) and\textbf{Mr. Hilary Kansiime} (IT assistant Kampala Water). I have worked under their supervision. They have guided me with alot of effort and time.
\par The co-operation is much indeed appreciated. I would also like to thank \textbf{National Water \& Sewerage Corporation} for allowing me to do my internship from their great Organization.
\par I also would like to thank all the respected officers and employees of \textbf{NWSC} for their continuous inspiration and support.
\par I am also very grateful to all of my lecturers and fellow friends for their encouragement and cooperation throughout my Internship and academic life.
\par Finally, I am forever grateful to my parents for their support and love.
\begin{flushright}
\textbf{Dinton.A.H}
\end{flushright}

\newpage
\textbf{ABSTRACT}\\
This report is about to explain what I learned during my internship period with National Water \& Sewerage Corporation, IT department,Kampala Branch. As the main purpose of internship is to learn by working in practical environment and to apply the knowledge acquired during the studies in a real world scenario in order to tackle problems using the knowledge and skills learned during the academic process.
\par This report is divided into four  sections. Section one will discuss about the background of NWSC. In Section two we will get the different activities I carried out during the Internship period. Section three the overall Internship Experience. The last section is conclusion and recommendations of the previous sections.
\par My training focused on various things like system administration, networking, hardware repair and maintenance. Some of the activities i did during my internship training include assembling and disassembling computer systems, adding computers to the domain, IP print services, software installation, Network maintenance and trouble shooting, providing user support,IT stock taking  and so many others.
\par The most important in an Internship program is that the student should spend their time in a true manner and with the spirit to learn practical orientation of theoretical study framework. This report is about my Internship that i have undergone at NWSC, IT department, Kampala Branch from June-August 2018.
\newpage
\tableofcontents
\thispagestyle{empty}
\cleardoublepage
\setcounter{page}{1}
\newpage
\begin{appendix}
\listoffigures
\newpage
\listoftables
\end{appendix}
\newpage
\textbf{List of Acronyms}\\
NWSC\dotfill  National Water \& Sewerage Corporation\\
IT\dotfill Information Technology\\
MD\dotfill Managing Director\\
IP\dotfill Internet Protocol\\
KW\dotfill Kampala Water\\
IREC\dotfill International Resource Center\\
DHCP\dotfill Dynamic Host Configuration Protocal\\
DNS\dotfill Domain Name System\\
FTP\dotfill File Transfer Protocol\\
IIS\dotfill Internet Information Services\\

\newpage
\textbf{Main Report}\\
\section{Introduction}\cite{r1}
\subsection{Background of NWSC}
National Water and Sewerage Corporation (NWSC) is a public utility company 100 percent owned by the Government of Uganda. The Corporation was established in 1972 under Decree No: 34. At its inception in 1972, the Corporation operated in three (3) major towns of Kampala, Jinja and Entebbe. These laws were revised in 1995 by the NWSC Statute and later on the statute was incorporated in the Laws of Uganda as CAP 317 (Laws of Uganda 2000). The primary aim of this was to revise the objectives, powers and structure of NWSC to enable the corporation operate and provide water \& sewerage services in areas entrusted to it on a sound commercial and viable basis.
\subsubsection{Vision}
"To be the leading customer Service Oriented utility in the world." 
\subsubsection{Mission}
"To sustainably and equitably provide cost effective quality water and sewerage services to the delight of all stakeholders while conserving the environment."
\subsubsection{Objectives}
To provide water and sewerage services in urban areas under its Jurisdiction.
\subsubsection{Core Values}
\textbf{P}rofessionalism\\ 
\textbf{R}eliability\\
\textbf{I}ntegrity\\
\textbf{I}nnovation\\
\textbf{T}eam work\\
\textbf{E}xcellence\\
\textbf{R}esult Oriented\\
\subsubsection{Operations/Services of NWSC}
\underline{\textbf{Water Quality}}\\ 
NWSC ensures that the water supplied up to the customer's meter is of a very high standard. Customers are requested to maintain the water quality within their premises. Internal plumbing systems, water heating systems and water storage facilities, such as overhead tanks at homes, can compromise the quality of drinking water, if not well maintained.\\
\underline{\textbf{Water \& Sewerage Services}}\\ 
NWSC ensures that the water  provided meets all regulatory standards they adhere to and distribute it equitably.\\
\underline{\textbf{Extension of Water \& Sewerage Services}}\\ 
NWSC endeavours to serve the populations of the areas in which they operate with clean, reliable and safe water.\par They make mains extensions of water services to areas meet the demand.\par They also provide Sewage collection, treatment and disposal services in accordance with the required environmental standards.\\  
\underline{\textbf{Service Connections}}\\ 
NWSC connects all applicants who meet their standard requirements in the areas where their services exist.\\
\underline{\textbf{Meter Reading}}\\
They read all meters every 30 days.Where this is not possible, a  reasonable estimate is determined using previous consumption trends of the last three consecutive readings.\\
\underline{\textbf{Billing and Bill Distribution}}\\
They maintain a 30 days billing cycle for all customers. The water and sewerage billing is based on the consumption as recorded from the meter. Where a meter is not read, an estimate based on the three correct previous readings is used. The invoice is produced and delivered on the day of meter reading where we have adopted the on spot billing system and where bills are printed in the office, the bill is delivered by the fifth month.\\
\underline{\textbf{Collection of Payments}}\\
Payments can be done by using the e-water payment system using mobile money and through the banks.\\
\underline{\textbf{Network Maintenance}}\\
NWSC carries out daily monitoring of their distribution network and deal with all identified leaks and bursts within 24 hours.\\
\underline{\textbf{Sewerage Treatment}}\\
NWSC provides sewerage collection, treatment and disposal services in accordance with the required environmental standards.\\
\underline{\textbf{Sewerage Disposal}}\\
The Corporation only disposes off treated effluent that meets regulatory and other stakeholder requirements and in so doing, they ensure safety their  customers, workers and the integrity of the receiving environment.     
\subsection{Structure of NWSC}
\subsubsection{Organizational structure}
NWSC has to thrive to emplace management system that is democratic, honest, inspiring, transparent, and highly participatory. The corporation's top management includes the Managing Director,Deputy MD Technical Services, Deputy MD Finance \& Corporate Strategy, Corporation Secretary, Director Engineering Services, Director Business \& Scientific Services, Director Commercial \& Customer Services, Director Internal Audit, Director Planning \& Capital Development, General Manager Kampala Water, Director Revenue,Finance \& Accounts.
\begin{figure}[h!]
\includegraphics[width=\linewidth]{/Users/Dintaine/Desktop/latex/figures/nwsc.png}
\caption{Organizational Structure and Work flow of NWSC}
\label{fig:structure}
\end{figure}
\subsubsection{IT Structure}
The top management of the IT department consists of the Senior Manager,Manager Systems Administrator,Manager Application Developer,Manager Infrastructure.
\begin{figure}[h!]
\includegraphics[width=\linewidth]{/Users/Dintaine/Desktop/latex/figures/Screenshot.png}
\caption{I.T Structure of NWSC}
\label{fig:IT structure}
\end{figure}
\subsection{Main Activities of NWSC and ongoing IT projects}
\subsubsection{Prepaid Metering System Project}
The prepaid metering system is a simple technology with three components; a card, an Internal Unit and a meter. Both the meter and the card reader will be installed at the premise. Each card is tailored to the meter and will be auto loaded with credit at NWSC approved vending points after which it is inserted into the Card reader Internal Unit to recharge the meter.
\par Before the credit runs out, the meter will produce an alert noise, if the card is not recharged the valve of the meter closes and water flow is stopped. The recharging points will be placed in different NWSC offices and will be accessible at all times.\\
\textbf{Benefits}
Encourages Water Consumption Control.\\
Shows Current credit and the corresponding water left.\\
Customer can see current and past consumption.\\
Identification of leaks.\\
Smart Card – Simplicity.\\
Emergency Water – Flexibility.

\section{Field Attachment Activities}
\section{The Overall Internship Experience}
\subsection{Inroduction}
This chapter talks about the  new skills gained,experiences as well as the challenges faced and how I managed to solve them during Internship. It also explains the relatedness of University’s taught programmes to the Field work, benefits derived and Career Motivation.
\subsection{Lessons and Experiences}
\begin{itemize}
\item I was able to learn how to work efficiently under minimal Supervision since I was given tasks to do when my supervisor was in serious meetings.
  \item \textbf{Work Ethics related Issue}:Internship is an opportunity to learn the skills and behaviors along with the work values that are required for success in the workplace. Workplace ethics are established codes of conduct that reflect the values of the organization or company where you are employed. I was able to possess the willingness to work hard for my supervisor during my internship period. In addition to working hard it is also important to work smart. This means I acquired the most efficient way to complete tasks and finding ways to save time while completing daily assignments. It’s also important to care about my job and complete all projects while maintaining a positive attitude.
\item \textbf{Practical Skill}:The Internship gave me  the opportunity to connect classroom theory with current industry challenges, and have exposure to the latest technologies. Opportunities to converse and interact with a large pool of talented experienced department members  provided a deeper insight tothe overall operation, as well as providing a valuable pool of resources to assist in completion of internship program. This internship program was exactly what I needed to nurture the lack of practical skills I had. I would acquire practical experience to complement the theoretical content of my studies.
\item \textbf{Theoretical Knowledge}:During my internship period at NWSC ,I was able to  upgrade my theoretical knowledge,from the courseunit  of  Computer Networks in theclassroom for example Microsoft windows server 2016, DNS ,DHCP ,IIS ,FTP and Active directory domain service.
  \item \textbf{Time Management}:  I learnt how to be punctual. NWSC is a busy organization and work had to be done as early as possible arrival time was 7:30am.
\item \textbf{Communication Skills} : I improved on my  communication skills in a formal and informal way. In the formal way as I had to communicate with my supervisor, staff and interacting with people while providing user support.
\item \textbf{Internet Research Skills} : I greatly improved my research skills as most of the things learnt i had to get more knowledge about them so I used a great deal of websites to acquire more knowledge by researching on them.
\item \textbf{Team Work} : I worked with fellow interns in teams on tasks assigned such as the IT stock taking exercise where we had to record the different devices owned by NWSC for example mice, keyboards, laptops,system units,switches,routers and so many others by recording the serial number, make, status and the user. At the end of the exercise we had to come up with a report. 
\item \textbf{Data Entry}: I improved my skills in data entry using software packages like Excel while analyzing data collected from stock taking exercise as well as creating different mail groups and entering data for example a mailgroup for Kampala Water Finance and Administration,KW branch engineers,KW commercial officers, KW commercial assistants, KW Branch Managers.
\item \textbf{Knowledge of the Organisation}: I gained knowledge on the operations of NWSC; and the overall staff conduct and guidelines through orientation which took place in the first week at NWSC IREC.
\item User Support.
\item Network printer Installation.
\item Software Installation.
\item Server Configuration and Management.
\item Microsoft Office Installation.
\item Anti-Virus Installation and Activation.
\item Operating System Installation.
\end{itemize}
\section{Conclusion and Recommendations}
\subsection{Conclusion}
\subsection{Recommendations}
\section{References}
\section{Appendix}
\begin{figure}[h!]
\includegraphics[width=\linewidth]{/Users/Dintaine/Desktop/latex/figures/systemunit.jpg}
\caption{An open System Unit showing the interior components after disassembling process}
\label{fig:System Unit}
\end{figure}
\begin{figure}[h!]
\includegraphics[width=\linewidth]{/Users/Dintaine/Desktop/latex/figures/server.jpg}
\caption{Me in the Server Room}
\label{fig:Server Room}
\end{figure}
\begin{figure}[h!]
\includegraphics[width=\linewidth]{/Users/Dintaine/Desktop/latex/figures/tools.jpg}
\caption{A tone Tracer and Cable Tester}
\label{fig:Tools}
\end{figure}
\begin{figure}[h!]
\includegraphics[width=\linewidth]{/Users/Dintaine/Desktop/latex/figures/blower.jpg}
\caption{Me using a blower to perform hardware maintenance(Removing dust from the interior components of the Server)}
\label{fig:Blower}
\end{figure}
\begin{figure}[h!]
\includegraphics[width=\linewidth]{/Users/Dintaine/Desktop/latex/figures/cartridge.jpg}
\caption{An IBM LTO Ultrium Data 6 Cartridge(Inserted into the Tape Library in the Server Room to improve on Data Storage}
\label{fig:Cartridge}
\end{figure}
\begin{figure}[h!]
\includegraphics[width=\linewidth]{/Users/Dintaine/Desktop/latex/figures/frends.jpg}
\caption{Me,my colleagues and Supervisor in the Server Room }
\label{fig:Friends}
\end{figure}





\bibliographystyle{apalike}
\bibliography{reftern1}
\end{document}
